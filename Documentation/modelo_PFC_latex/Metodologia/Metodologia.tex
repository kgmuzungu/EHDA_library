\chapter{Metodologia}
\label{chap:metodologia}

Este capítulo deve descrever os métodos utilizados no projeto. As teorias e ferramentas utilizadas, assim como as ações executadas, devem ser escritas de forma clara e precisa, sem deixar espaço para ambiguidades. Tenha em mente que \textbf{o objetivo aqui é dar credibilidade ao seu trabalho e permitir que ele possa ser reproduzido por quem tenha interesse (algumas vezes por você mesmo!)}. 

É natural que este capítulo contenha uma revisão bibliográfica mais detalhada das técnicas utilizadas. O ideal é que o capítulo seja auto-contido, de modo que o leitor não necessite recorrer às referências para entender seu trabalho ou reproduzi-lo.

Ao apresentar os métodos, será importante justificar devidamente as opções tomadas e discutir as alternativas disponíveis e os critérios que o levaram à escolha do método.

No restante do capítulo, apresentamos um pequeno manual de escrita técnica.

\section{Técnica 1 - Organização do texto}
\label{sec:metodo1}

Procure \textbf{definir a estrutura} de seu texto \textbf{antes de começar a escrever}. Isso evitará que as ideias apareçam foram de ordem. A falta de organização é uma das características mais comuns de textos difíceis de ler.

Nunca invoque conceitos ou objetos antes de defini-los. Procure fornecer o máximo de informação \emph{a priori} possível. Evite referências para frente no texto. 

Para evitar que o leitor se perca, antes de iniciar um trecho longo de texto que contenha ideias diversas, \textbf{adiante para o leitor o que será feito}, aonde se quer chegar. Ao fim do trecho, declare explicitamente que chegou à conclusão que desejava. Esta técnica vale especialmente para seções e capítulos do texto. 

\section{Técnica 2 - Estilo}
\label{sec:estilo}

O texto técnico deve ser preciso, claro, sem ambiguidades, objetivo e conciso. Mesmo com esse rigor, não deixe de prender a atenção do leitor. Lembre-se de que o início e o fim de cada parágrafo, de cada frase, são as partes mais importantes e onde você deve colocar maior ênfase.


\subsection{Precisão}

Use as palavras corretas! Não escreva redondo se é esférico. Não escreva igual se é aproximado. Não escreva sistema se é função de transferência. Não escreva grande, pequeno, máximo, mínimo, ótimo, se não há uma escala que defina tais conceitos.

Prefira usar números que deem uma noção de escala a usar adjetivos difíceis de precisar.

Seja específico e evite generalidades. Por exemplo, em vez de dizer ``O processo 1 é um processo de alto custo'', devemos ser mais precisos com relação à natureza do custo e dizer ``O processo 1 possui custo operacional elevado com relação a processos tradicionais tais como o processo 2''.

\subsection{Clareza}

Não deixe margem para dúvidas! Para reduzir a possibilidade de confusão na leitura do seu texto, \textbf{evite frases longas}. Nunca escreva frases com mais de $20$ palavras. Procure escrever na média $12$ palavras por frase. Evite palavras compridas ou que possam ser consideradas difíceis.

Não deixe ideias subentendidas! Não assuma que seu leitor pensa como você e vai saber do que você está falando. Explicite todos os passos de seu raciocínio lógico. Pular passos de raciocínio é uma das causas mais comuns de confusão na leitura. Não seja preguiçoso neste quesito.

Novamente, defina todos os conceitos antes de fazer uso deles.

Não deixe o sujeito verbal subentendido. Para evitar ambiguidades, tome cuidado ao usar pronomes como ``esse, este, isso, ele''. É preferível repetir um nome a deixar margem para dúvidas. Por exemplo, após a primeira frase deste parágrafo poderíamos dizer de forma não tão clara: ``Esta é uma causa de confusão na escrita.''; ou poderíamos dizer de forma mais clara: ``Essa falta de informação é uma causa de confusão na escrita.''

Não há problemas em haver repetições em textos técnicos. O mais importante é a clareza, a inexistência de ambiguidades, não a beleza. Se acha que uma determinada frase pode ser confusa ou se alguma ideia parece difícil de explicar, repita a mesma ideia em outras palavras ou, melhor ainda, dê um exemplo.

Por fim, domine o significado das palavras utilizadas para evitar ambiguidades. Se uma palavra pode ter mais de um significado em determinada frase, tente trocá-la por outra que não deixe dúvidas. Mais importante ainda, não use palavras sobre cuja definição você não tem certeza.

\subsection{Ojetividade}

O texto técnico deve ser direto ao ponto, sem rodeios, e livre de opiniões. Evite apartes, evite divagações e discussões irrelevantes com relação ao tema principal.

\textbf{Nunca use hipérboles}. Cuidado com termos como ``otimizar, muito grande, enorme, muito bom''.  Usar números sempre que possível para quantificar conceitos, sejam eles precisos ou apenas uma estimativa. Em especial deve-se tomar cuidado com as palavras ``muito'' e ``muitos''.

O tom deve ser impessoal, apresentando apenas fatos e não opiniões. Em português, a voz passiva é um instrumento comum para se obter um tom impessoal. Contudo, sempre que possível, tente usar a voz ativa ou a voz passiva sintética para deixar o texto mais fluido e claro.

Palavras abstratas deixam a escrita vaga. Prefira palavras concretas, fortes, isto é, palavras que tenham um único significado. Sempre que possível substitua múltiplas palavras por uma única. Por exemplo, na frase ``A máquina processou as amostras'' temos duas palavras genéricas: máquina e processar. A mesma ideia seria mais objetiva se expressa como ``A centrífuga girou as amostras''.

Controle seu tom. \textbf{Não use linguagem coloquial. Não use clichês}. Não use de arrogância: ``obviamente, como é sabido, é claro que''. 

\subsection{Concisão}

O texto técnico deve ser tão curto quanto possível, sem prejuízo de sua clareza e precisão. Não use palavras a mais, não inclua expressões irrelevantes ou supérfluas. Não escreva simplesmente para encher as páginas. Se um assunto não é relevante para a compreensão do seu trabalho, corte-o, mova-o para um apêndice ou aponte uma referência.

Evite redundâncias como ``opinião pessoal'' e ``garantia absoluta''.

\section{Vícios comuns}

\begin{itemize}
\item Estrangeirismos: ``\emph{performance}'', ``\emph{mutatis mutandis}''. Se necessário, coloque o termo estrangeiro em itálico e dê uma tradução entre parênteses.

\item Usar linguagem rebuscada. Lembre-se de que a linguagem técnica deve ser clara, precisa e objetiva.

\item Alternar o tempo verbal entre passado e presente. Seja consistente e escreva apenas no passado ou no presente.

\item Iniciar uma frase com verbo porque o sujeito está implícito na frase anterior. Deixe sempre o sujeito explícito.

\item Inserir referências no meio de uma frase. Isso quebra o fluxo da leitura. Coloque a citação no fim da frase.

\item \textbf{Usar siglas sem antes defini-las}. Sempre escreva a sigla por extenso \emph{na primeira vez} que ela aparece no texto. Por exemplo, ``este é um texto de Projeto Final de Curso (PFC)''. 

\item Escrever ``como sabemos'', ``como é sabido'', ``por razões óbvias'', ``é evidente que'', ``talvez seja verdade que''. Estes termos significam apenas que você não sabe explicar o que está afirmando.

\item Usar definições \emph{por exemplo}, isto é, apresentar a definição de um conceito ou termo por meio de um exemplo. Primeiro defina o conceito, depois exemplifique.

\end{itemize}

\section{Boas práticas}

\begin{itemize}
\item Sempre que introduzir novos termos e conceitos, destaque-os em itálico para que o leitor saiba que se trata de uma definição e para que ele ache a definição com facilidade quando for necessário rastreá-la no texto.
\item Releia o que escreve a cada parágrafo. Releia rapidamente cada seção para verificar o encadeamento de ideias.
\item Corte palavras sempre que possível.
\item Nunca assuma que o leitor entenderá o que você escrever. Esforce-se para fazê-lo entender.
\item Use um corretor ortográfico.
\item Peça que alguém leia seu texto.
\end{itemize}


\section{Uso de referências}
\label{sec:comocitar}

Uma afirmação incluída num texto técnico se enquadra em um de três casos: a) seu conteúdo é de conhecimento geral dentro da área do texto; b) seu conteúdo é original e resulta do trabalho do autor; c) seu conteúdo tem origem em outro trabalho (ainda que seja do mesmo autor, não é original).

Toda afirmação enquadrada na categoria c) deve ser acompanhada de uma citação. Para afirmações da categoria b), deixe claro que se trata de ideia original do autor. Isso evita que o leitor se confunda ao pensar que possa se tratar de a) ou c).

No caso a), pode ser um tanto mais sutil determinar o que deve ser de conhecimento geral. Procure evitar o excesso de citações. Se todo um parágrafo se baseia em ideias de uma certa referência, em geral basta que ela seja citada no seu início.

\textbf{Atenção:} Não copie frases das referências usadas. Reescreva as ideias com suas próprias palavras e não deixe de citar a fonte. Se for necessário manter as palavras do original, cite a frase entre aspas e em itálico.

\section{Honestidade e Plágio}

Seja honesto ao escrever. \textbf{Não ``maqueie'' seus resultados}. Não apresente apenas as melhores amostras dos seus resultados para dar a impressão de que foi bem sucedido. Não escreva aquilo que não entende. Não escreva de forma vaga para mascarar o fato de que não entende algo.

Dê crédito a quem o merece. Inclua referências sempre que usar o trabalho de outros. Seja sempre claro para não dar a impressão de que fez algo feito por outrem. \textbf{Não assuma crédito pelo trabalho dos outros}. Isso pode ter implicações legais e acadêmicas.

\section{Cuidado com a gramática}

\begin{itemize}

\item Erros gramaticais deixam uma \textbf{impressão ruim} e podem alterar a disposição do leitor para com o texto.  

\item Use a vírgula corretamente. Seu uso incorreto confunde o leitor. Nunca use a vírgula porque acha que a frase precisa de uma pausa. Não separe sujeito e verbo. Quando há inversão da ordem natural em uma frase, \emph{toda} a parte movida deve estar entre vírgulas.

\item Cuidado com a concordância da voz passiva sintética. Lembre-se de que o correto é ``Vendem-se ovos'' e não ``Vende-se ovos''.

\item Use a crase corretamente. Faça sempre o exercício de substituir o nome que sucede o ``à'' por um nome no masculino. Se o correto for usar ``ao'' com o nome masculino, então deve haver crase. Nunca use crase antes de nomes no masculino!

\item Escreva ``em que'' em vez de ``onde'' quando não houver indicação de lugar.

\item Não se escreve ``o fato \textbf{dela} ser'' ou ``a razão \textbf{do} texto ser escrito''. Escreve-se ``o fato \textbf{de ela} ser'' e ``a razão \textbf{de o} texto ser escrito''.

\item Releia o que escreve e na dúvida busque ajuda.

\end{itemize}

\section{Resumo do Capítulo}
\label{sec:metodo1b}

Tente não terminar de forma abrupta. Se for escrever algo aqui, não seja genérico!


\clearpage