\section{Conclusion}

In this step of the project the python code was upgraded to optimize the data acquiring, processing and saving. Also 
started structuring the software to fit our control model. Initial experiments was made in order to evaluate the algorithm 
performance and intuitivity. With the data collect in those first experiments we can notice that:

- The algorithm is not capable of separating cone jet mode and multi jet mode.

- The algorithm is not capable of sepaating intermittend and dripping.

- It has noise in the current values which are not discovered yet.

As the method proposed by Sjaak\cite*[]{Sjaaks} classifies each spraying mode using deviation/mean and mean/median relations,
we can se in Figure 14 that the colors are well separated by certain deviation/mean range.

For next steps experiments will be done varying the step size and step time of the power supply to get a better undertanding in how
the change in dynamics behave. It is already known that it needs an adapting time to stabilize in a certain spraying mode. The next step
will try to measure and understand this time. It is also known that the system has an histeses when changing between spraying modes. This is also
another target for further studies.


