\addcontentsline{toc}{chapter}{Resumo}

\begin{center}
\huge{{\bf Resumo}}
\vspace{2cm}
\end{center}

A Atomização Eletro-hidrodinâmica (EHDA), também chamada de electrospray, é uma técnica de atomização de líquidos que produz gotículas de dimensão micro e nanométricas, utilizando campos elétricos elevados (kV/cm).
De acordo com Cloupeau e Prunet-Foch\cite{prunet} (1994) , os electrosprays podem gerar gotículas em diferentes dinâmicas, que os autores denominaram "modos de electrospray". 
 Esses modos podem ser ajustados variando a força do campo elétrico e a vazão de líquido, mas também dependem da geometria e das propriedades físico químicas sistema. 
 Em seu trabalho, os autores propuseram quatro possíveis modos EHDA: gotejamento, intermitente, jato cone e jato multi, que são geralmente distinguíveis visualmente.
 Verdoold et al.\cite{Sjaaks} (2014) sugeriram recentemente uma abordagem de classificação com base no comportamento da corrente elétrica do processo de electrospray.

Este projeto desenvolve um método de controle em malha fechada para dispositivos EHDA que utiliza classificação de modo de spray em tempo real com base em corrente elétrica (portanto, não visual). 
O projeto de sistema de electrospray é totalmente automático, onde todos os periféricos, como fonte de alimentação HV e bomba de seringa, são controlados por um computador que executa suas rotinas. 
O sistema classifica a dinâmica do modo de spray usando dados de corrente em tempo real e altera os parâmetros operacionais do EHDA, como vazão do líquido e tensão aplicada, para alcançar e manter o modo de spray escolhido. 
Os modos de electrospray são validados em tempo real usando uma câmera de alta velocidade.
Em comparação com as abordagens manuais convencionais, o algoritmo de controle implementado alcança maior precisão e menor tempo de transição. 
Portanto, um sistema EHDA completamente autônomo abre portas para potenciais aplicações industriais. 
Além disso, o uso do sinal de corrente elétrica será útil para estudar mais a fundo os processos de electrospray, levando a um melhor controle na geração de gotículas (frequência, tamanho e carga). 
A incorporação de Aprendizado de Máquina para melhorar a categorização de modos será um desenvolvimento futuro.

\clearpage
\thispagestyle{empty}
\cleardoublepage

% No Resumo, em uma única página, em no máximo dois parágrafos, você explicita os seguintes itens: objetivos do projeto e descrição sucinta do local onde ele foi desenvolvido; metodologia utilizada; e resultados alcançados. Leitores experientes decidem se prosseguirão para a leitura do texto completo após lerem o resumo, a conclusão e a introdução. Por isso nestes lugares você deve colocar um esforço maior de convencimento. Além disso, a linguagem utilizada deve ser acessível a leitores com pouca familiaridade com a área, limitando-se o uso de jargões.
 
% \begin{sloppypar}
% Este novo parágrafo serve para mostrar que ao pular uma ou mais linhas no texto do arquivo .tex, o \TeX\ entende que você está iniciando outro parágrafo. O comando \textsf{sloppypar} força o texto a não ultrapassar as margens. Só deve ser usado se este problema ocorrer.
% \end{sloppypar}
