\chapter{Introdução}
\label{chap:intro} %este label será usado para referenciar este capítulo

As primeiras frases têm a missão de prender a atenção do leitor e por isso são as mais importantes do texto. Diga o quanto antes o que você fez e quais são os resultados alcançados. Ao terminar de ler a introdução o leitor tomará uma nova decisão de se vale a pena ou não continuar lendo o texto. Capte a atenção do leitor bem aqui.

A comunicação escrita é considerada umas das cinco habilidades mais importantes por profissionais de engenharia e um engenheiro passa em média mais de $25$\% do seu tempo escrevendo \cite{eggert2002response,spretnak1982survey}. Uma quantidade similar de tempo é gasta na escrita de correspondência e de relatórios técnicos \cite{cunningham2012perceptions}. Dessa forma, encare a escrita do seu projeto como um treinamento nessa importante habilidade.

Neste texto você encontrará não apenas uma estrutura para escrever seu trabalho em \LaTeX, mas também um pequeno manual de boas práticas na escrita técnica. Leia com atenção e coloque as sugestões em prática à medida que preenche o texto com o conteúdo do seu próprio projeto. Também será apresentado um número de vícios de escrita comumente encontrados nas monografias de alunos. 

A seguir está a estrutura de organização sugerida pelo colegiado do curso. Note que ela não é necessariamente a melhor para contar a história do seu projeto. Você pode por exemplo preferir usar títulos mais pertinentes ao seu contexto. Contudo, o seu texto deve conter cada um dos pontos a seguir.

\section{Motivação e Justificativa}
\label{sec:motivacao}

Argumente sobre a importância do projeto desenvolvido usando uma visão de alto nível, sem entrar em detalhes. Contextualize seu projeto dentro do local de execução ou da literatura e explique como ele é necessário ou inovador. É possível fazer uma breve revisão bibliográfica, confrontando seu trabalho com outras referências bibliográficas para mostrar a sua contribuição. No quesito contribuição, é muito importante deixar claro o tempo todo que partes do projetos foram executadas por outros e que partes foram executadas por você. Caso contrário, corre-se o risco de inadvertidademente tomar crédito pelo trabalho de outrem, o que pode ter implicações legais. 

\section{Objetivos do Projeto}
\label{sec:objetivos}

Tendo em vista o exposto acima, este projeto tem por objetivos:

\begin{enumerate}[a)]
\item Item 1;
\item Item 2; 
\item Etc.     
\end{enumerate}

O conteúdo desta seção pode se sobrepor um pouco com o da seção anterior, podendo ela ser um sumário dos pontos expostos anteriormente. A escolha do título da seção talvez seja mais apropriada para a fase de proposta do projeto. Afinal, nesta fase se conhecem os objetivos e não os resultados. Por outro lado, fará pouco sentido discutir objetivos quando o projeto está finalizado, especialmente se tais objetivos não foram alcançados. 


\section{Local de Realização}
\label{sec:empresa}

Faça uma breve apresentação da empresa ou laboratório onde o projeto é desenvolvido. Fale um pouco da história da empresa, do mercado em que atua, da sua organização, do departamento em que está inserido o projeto, etc. Descreva também o seu vínculo com a empresa. 


\section{Estrutura da Monografia}
\label{sec:organizacao}

O trabalho está dividido em quatro capítulos. Este capítulo apresentou uma introdução ao projeto a ser descrito nesta monografia e a empresa onde o trabalho foi realizado. O Capítulo 2 descreve os princípios básicos de um sistema ... (sistema onde se insere o trabalho) e abrange todos os conceitos necessários para um melhor entendimento do projeto. O Capítulo 3 aborda a metodologia de desenvolvimento, seguida pela implementação dos .... No Capítulo 4 tem-se a conclusão da  monografia e algumas sugestões e dificuldades encontradas na realização do projeto.


\clearpage