\chapter{Literature Review}
\label{chap:lit_review}


\section{EHDA}
\label{sec:ehda_resume}

The electrospraying of liquids herein is referred to as electrohydrodynamic atomization (EHDA). The atomization by primarily electrical (electro) forces of a liquid (hydro) that is moving (dynamic) during the atomization captures the essence of the phenomena.\cite{Grace}
That motion applies to the liquid certain velocity that is not enough to create the spray alone. Therefore, the eletric field itself is the responsible for the spraying dynamics.\cite{prunet}

The stable balance between the capilary and field forces on the liquid suggest a \emph{quasi static} dynamics.
For this reason with a controlled enviroment we can reach a certain stable spraying mode as can be seen in the Figure \ref{fig:ehda_setup_ex2}.

\begin{figure}[H]
  \centering
  \resizebox{80mm}{!}{\includegraphics{Figuras/ehda_ex_2.png}}
  \caption{EHDA physical concept \cite{Gabriel}}
  \label{fig:ehda_setup_ex2}
\end{figure}


\section{Spraying modes}
\label{sec:spraying_modes_subsec}

Since 1915 with his pioneering work in EHDA, Zeleny observed several functioning modes with very different characteristics.
Years later the same phenomena was noticed by other scientists but the classification of these modes were still not well defined by the community.
For that Cloupeau and Prunet-Foch proposed spray mode classifications based in what they have seen experimentally and it's still being used as basis for EHDA researchs.\cite{prunet}

The Figures \ref{fig:dripping_camera_example}, \ref{fig:cone_camera_jet_example}, \ref{fig:multi_camera_jet_example} shows 3 spraying dynamics that we are most interesting in this project. 

\begin{multicols}{3}

  \begin{figure}[H]
      \center
      \includegraphics[width=3cm]{Figuras/drippingexample.png}
      \label{fig:dripping_camera_example}
      \caption{Dripping}
  \end{figure}


  \begin{figure}[H]
      \center
      \includegraphics[width=3cm]{Figuras/conejetexample.png}
      \label{fig:cone_camera_jet_example}
      \caption{Cone Jet}
  \end{figure}


  \begin{figure}[H]
      \center
      \includegraphics[width=3cm]{Figuras/multijetexample.png}
      \label{fig:multi_camera_jet_example}
      \caption{Multi Jet}
  \end{figure}

\end{multicols}


Through the various classifications and subclassifications of spraying defined in literature we are going to aggregate some of them and separate beetween 5 modes as shown above in order of growing electric potential:

\subsection{Dripping}
\label{subsec:dripping}

Dripping mode happens when the eletric field applied is not enough to change the meniscus shape, phenomena called field enhanced dripping.
In that situation the liquid droplet has, in general, size bigger than the capilary and low frequency intervals between each drop.

\begin{figure}[H]
  \center
  \includegraphics[width=3cm]{Figuras/19:03/drip_example.png}
  \label{fig:drip_example}
  \caption{Dripping}
\end{figure}

\subsection{Intermittent}
\label{subsec:Intermittent}

Intermittent mode is defined when the eletric field forces starts to have a considerable effect in the meniscus and droplet formation. 
In this mode the droplet size is smaller than the nozzle, phenomena called microdripping, and the dripping frequency increases with the increasing of the field applied.

\begin{multicols}{3}

  \begin{figure}[H]
      \center
      \includegraphics[width=3cm]{Figuras/19:03/intermittent_example.png}
      \label{fig:intermittent_example}
      \caption{Intermittent}
  \end{figure}


  \begin{figure}[H]
      \center
      \includegraphics[width=2cm]{Figuras/19:03/intermittent2_example.png}
      \label{fig:intermittent2_example}
      \caption{Intermittent}
  \end{figure}


  \begin{figure}[H]
      \center
      \includegraphics[width=2cm]{Figuras/19:03/microdripping_example.png}
      \label{fig:microdripping_example}
      \caption{Microdripping}
  \end{figure}

\end{multicols}

\subsection{Cone Jet}
\label{subsec:Cone Jet}

Taylor (1964) was the first to demonstrate that electrostatic pressure and capillary pressure can be balanced at any point on the surface of a liquid cone. 

\subsection{Multi Jet}
\label{subsec:Multi Jet}

\subsection{Corona sparks}
\label{subsec:Corona sparks}


\section{Non-visual classification}
\label{sec:non-visual-classification}

Since the beginning of EHDA until today the researchs are being conducted manually with visual classification of the spraying mode using a camera or even by naked eyes.
It is recommended to use a high speed (HS) camera because some dripping or intermittent states can be in a high frequency and be wrongly noticed as a stable condition.
The setup in \ref{fig:ehda_setup} shows the most common setup used for EHDA researchers.

\begin{figure}[H]
  \centering
  \resizebox{150mm}{!}{\includegraphics{Figuras/system_setup.png}}
  \caption{EHDA experiment setup \cite{Luewton}}
  \label{fig:ehda_setup}
\end{figure}


Therefore some researchs were made about the classification of the spraying mode measuring the current flowing through the nozzle to plate\cite{Sjaaks}\cite{Chen_Pui}. That current signal holds a lot o information 
about the dynamics that is happening with the liquid. In \ref{fig:microdripping_current_pic} we can see an example of that. It's clear the two droplets generated in this time frame.


\begin{figure}[H]
    \centering
    \resizebox{150mm}{!}{\includegraphics{Figuras/report2/img2.png}}
    \caption{Current measurement sample of a micro-dripping spraying mode. This graph represents 0.5s sample. The sampling frequency is 100kHz. Hence we have 50000 current values.}
    \label{fig:microdripping_current_pic}
  \end{figure}


\clearpage