\chapter{Introduction}
\label{chap:intro} 
%este label será usado para referenciar este capítulo

Electrohydrodynamic Atomization (EHDA) is a way to disintegrate a liquid into droplets by exposing it to a strong electric field.\cite{prunet}
The balance between capillary forces and the eletric field on the charged liquid defines the spraying dynamics and droplet size.
The electric current transported by the spray reveals characteristic shapes for different spray modes.
Signal processing techniques can allow a non-visual classification of the spray mode based on the electric current shape.\cite{Sjaaks}
The spray process imposes noise and random sequences on the measured signal making its classification not a trivial task. 
Industrial applications demand automated stabilization of a spray mode. 
This can be achieved by a closed-loop control system. 
This project is about to develop an application that can classify what dynamics the EHDA experiment is current in and control the variables to stabilize in the desired mode. 


\section{Motivation and Justify}
\label{sec:motivacao}

% Argumente sobre a importância do projeto desenvolvido usando uma visão de alto nível, sem entrar em detalhes. Contextualize seu projeto dentro do local de execução ou da literatura e explique como ele é necessário ou inovador. É possível fazer uma breve revisão bibliográfica, confrontando seu trabalho com outras referências bibliográficas para mostrar a sua contribuição. No quesito contribuição, é muito importante deixar claro o tempo todo que partes do projetos foram executadas por outros e que partes foram executadas por você. Caso contrário, corre-se o risco de inadvertidademente tomar crédito pelo trabalho de outrem, o que pode ter implicações legais. 

EHDA research has contributed as an important tool for the development of technology. 
The advantage of using EHDA is precision and uniform size and shape of droplets creation. Specially in certain spraying modes.

Although there are applications of EHDA in industry, the stabilization of the conical jet spray mode is mostly done empirically and based on average current measurements.

The flowing electrical current carried by the spray reveals characteristic shapes for different atomization modes.
Signal processing techniques may allow non-visual classification of spray mode based on electrical current form. 

The spraying process imposes noise and random sequences on the measured signal making it sorting is not a trivial task.
Industrial applications require automated stabilization of a spray mode. 
This can be achieved by a closed system loop control system. 
Automated spray mode sorting is a crucial part of a control system, as well as the development of an appropriate control algorithm.

\section{Project Goals}
\label{sec:goals}

This project aims to give continuity to the previous student work\cite{Monica}, 
Mônica, who foccused in detecting undesired discharges (sparks) in the system accusing high eletric potential. 
For that, she developed a python software routine to connect most of the peripherals and analysed the current data.
Her work corroborate the validation of this project motivating its development and optimization.

Above is shown the main goals listed:

\begin{enumerate}[a)]
\item multipurpose applications both scientific and industrial application.
\item Fully automated and intuitive system for EHDA research and industrial application.
\item Real Time non-visual classification of spraying modes.
\item Control and stabilization on a desired spraying mode.   
\item System Portability and versatility.
\end{enumerate}

\section{People envolved}
\label{sec:companies}

% Faça uma breve apresentação da empresa ou laboratório onde o projeto é desenvolvido. Fale um pouco da história da empresa, do mercado em que atua, da sua organização, do departamento em que está inserido o projeto, etc. Descreva também o seu vínculo com a empresa. 

The NHL Stenden Water Technology group has been involved in previous projects that have successfully implemented automated signal processing techniques, resulting in highly ranked outcomes. 
However, further research is required to enhance the accuracy of the classification algorithm. In order to achieve this, the Water Technology Group at NHL Stenden University of Applied Sciences, in collaboration with Dutch companies, is conducting extensive research and implement appropriate classification algorithms. The aim is to combine analytical capabilities with infrastructure knowledge and availability to achieve optimal results.
As a student from UFMG, I am now actively involved in this research project to improve the automation usability, classification accuracy and system stabilization with signal processing techniques.

\section{Document Structure}
\label{sec:doc_struct}

