\chapter{Introduction}
\label{chap:intro} 

Electrohydrodynamic Atomization (EHDA) is a way to disintegrate a liquid into droplets by exposing it to a strong electric field.\cite{prunet}
The balance between forces on the charged liquid meniscus defines the electrospraying dynamics and droplet size.
The electric current transported by the spray reveals characteristic shapes for different spray modes.
Signal processing techniques can allow a non-visual classification of the spray mode based on the electric current shape.\cite{Sjaaks}
The spray process imposes noise and random sequences on the measured signal making its classification a non-trivial task. 
Industrial applications demand automated stabilization of a spray mode. 
This can be achieved by a closed-loop control system. 
This project targets the development of a system which allows real time acquisition of the electrospray current, treat and process its values, define in which specific mode the electrospray is running and control the necessary hardware to change and/or stabilize the spray at a desired mode.


\section{Motivation and Justify}
\label{sec:motivation}

EHDA research has contributed as an important tool for the development of technology. 
The advantage of using electrospray is precision and uniform size and shape of droplets creation, specially in certain spraying modes. 
Although there are many applications of EHDA in industry, the stabilization of the conical jet spray mode is mostly done empirically and based on average current measurements.

Many applications such as gas odorization, spray coating and pharmaceutical industries require automated stabilization of a spray mode. 
This can be achieved by a closed system loop control system. 
An automated spray mode classification is a crucial part of the control system to work, as well as the development of an appropriate control algorithm.

\section{Project Goals}
\label{sec:goals}

This project aims to give continuity to the previous student work\cite{Monica}, who focused in detecting undesired discharges (sparks) in the system accusing high electric potential. 
For that, it was developed a python software routine to connect most of the peripherals and analyzed the current data.
Her work corroborates the validation of this project motivating its continuation on development and optimization.

Bellow are shown the main goals listed:

\begin{enumerate}[]
\item Multipurpose applications for both scientific and industrial approaches.
\item Fully automated and intuitive system for EHDA.
\item Real time non-visual classification of electrospraying modes.
\item Control and stabilization on a desired spray. 
\item System portability and versatility.
\end{enumerate}


\section{People involved}
\label{sec:companies}

The NHL Stenden Water Technology group has been involved in previous projects that have successfully implemented automated signal processing techniques, resulting in highly ranked outcomes. 
However, further research is required to enhance the accuracy of the classification algorithm. 

In order to achieve this, the Water Technology Group at NHL Stenden University of Applied Sciences, in collaboration with Dutch companies, is conducting extensive research and implement appropriate classification algorithms. The aim is to combine analytical capabilities with infrastructure knowledge and availability to achieve optimal results.

As a student from UFMG, I am now actively involved in this research project to improve the automation usability, classification accuracy and system stabilization with signal processing techniques.

\section{Document Structure}
\label{sec:doc_struct}

This document is divided in 6 chapters. 

Chapter \ref{chap:intro} provides an introduction to EHDA and explains the inspiration behind the project.

Chapter \ref{chap:lit_review} delves into the literature concepts and knowledge that were utilized in this project.

Chapter \ref{chap:system_description} describes the system that was implemented to make this project work, along with the instruments and models used to apply the methodology.

Chapter \ref{chap:Methodology} presents the methodology applied in experiments to validate the system.

In Chapter \ref{chap:Results}, it is presented results and engage in discussions comparing the knowledge acquired through literature review.

Finally, in Chapter \ref{chap:conclusion}, this document is concluded with discussions about the goals achieved and areas for potential optimization.