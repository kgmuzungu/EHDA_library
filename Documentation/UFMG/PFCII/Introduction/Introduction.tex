\chapter{Introduction}
\label{chap:intro} 
%este label será usado para referenciar este capítulo

Electrohydrodynamic Atomization (EHDA) is a way to disintegrate a liquid into droplets by exposing it to a strong electric field.\cite{prunet}
The balance beetween capilary forces and the eletric field on the charged liquid defines the spraying dynamics and droplet size.
The electric current transported by the spray reveals characteristic shapes for different spray modes.
Signal processing techniques can allow a non-visual classification of the spray mode based on the electric current shape.\cite{Sjaaks}
The spray process imposes noise and random sequences on the measured signal making its classification not a trivial task. 
Industrial applications demand automated stabilization of a spray mode. 
This can be achieved by a closed-loop control system. 
This project is about to develop an application that can classify what dynamics the EHDA experiment is current in and control the variables to stabilize in the desired mode. 


\section{Motivation and Justify}
\label{sec:motivacao}

% Argumente sobre a importância do projeto desenvolvido usando uma visão de alto nível, sem entrar em detalhes. Contextualize seu projeto dentro do local de execução ou da literatura e explique como ele é necessário ou inovador. É possível fazer uma breve revisão bibliográfica, confrontando seu trabalho com outras referências bibliográficas para mostrar a sua contribuição. No quesito contribuição, é muito importante deixar claro o tempo todo que partes do projetos foram executadas por outros e que partes foram executadas por você. Caso contrário, corre-se o risco de inadvertidademente tomar crédito pelo trabalho de outrem, o que pode ter implicações legais. 

The biggest motivation to use EHDA is to have a more precise and uniform size and shape of droplets creation.


EHDA research has contributed as an important tool for the development of technology. 
Although there are applications of EHDA in industry, the stabilization of the conical jet spray mode is done empirically and based on average current measurements.
The flowing electrical current carried by the spray reveals characteristic shapes for different atomization modes.
These forms cannot simply be summarized by their average value. In figure one we can see an example of cone-jet electrospray mode.

Figure 1: example of EHDA

Signal processing techniques may allow non-visual classification of spray mode based on electrical current form. 
The spraying process imposes noise and random sequences on the measured signal making it sorting is not a trivial task.
Industrial applications require automated stabilization of a spray mode. 
This can be achieved by a closed system loop control system. 
Automated spray mode sorting is a crucial part of a control system, as well as the development of an appropriate control algorithm.

\section{Project Goals}
\label{sec:goals}

In view of the above, this project aims to:

\begin{enumerate}[a)]
\item multipurpose applications both scientific and industrial application.
\item Fully automated and intuitive system for EHDA research and industrial application.
\item Real Time non-visual classification.    
\item Control and stabilization of desired spraying modes.   
\item System Portability.
\end{enumerate}

% The content of this section may overlap somewhat with that of the previous section, and it may be a summary of the points made above. 
% The choice of section title may be more appropriate for the project proposal phase. 
% After all, at this stage the objectives are known and not the results. 
% On the other hand, it makes little sense to discuss objectives when the project is finished, especially if those objectives have not been achieved.

\section{People envolved}
\label{sec:companies}

% Faça uma breve apresentação da empresa ou laboratório onde o projeto é desenvolvido. Fale um pouco da história da empresa, do mercado em que atua, da sua organização, do departamento em que está inserido o projeto, etc. Descreva também o seu vínculo com a empresa. 

Signal processing implementations from previous projects from the NHL Stenden Water Technology group are showing good ranking results. 
More research is necessary to improve classification accuracy and research and implementation of proper classification algorithm. 
Because of this, the work will be done by the Water Technology Group at NHL Stenden University of Applied Sciences and in combination with Dutch companies to combine analysis possibilities with knowledge and availability of infrastructure.
