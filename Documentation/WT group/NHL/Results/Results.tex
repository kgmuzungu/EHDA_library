\chapter{Results}
\label{chap:Results}

\begin{figure}[H]
    \centering
    \resizebox{170mm}{!}{\includegraphics{Figuras/19:03/axs2.png}}
    \caption{Cone Jet Classification}
    \label{fig:multi_class_exp}
\end{figure}


\section{Classification}
\label{sec:classification_results}


\begin{figure}[H]
    \center
    \includegraphics[width=12cm]{Figuras/19:03/voltage_step.png}
    \caption{ exp-26-01-2 (V x Q)}
\end{figure}


\begin{figure}[H]
    \center
    \includegraphics[width=12cm]{Figuras/19:03/raw-data-example.png}
    \caption{ exp-26-01-2 (V x Q)}
\end{figure}


\begin{figure}[H]
    \center
    \includegraphics[width=12cm]{Figuras/19:03/classified-data-example.png}
    \caption{ exp-26-01-2 (V x Q)}
\end{figure}


\section{Map Sequence}
\label{sec:map_results}



    \subsection{Manual experiments}


    For better understand the effects of both voltage and flowrate in the spraying dinamycs manual experiments were made.
    Also in order to find the stability region of cone jet mode for the liquid and setup used.



    \begin{figure}[H]
        \center
        \includegraphics[width=12cm]{Figuras/report3/exp26-01-2.png}
        \caption{ exp-26-01-2 (V x Q)}
    \end{figure}


    \begin{figure}[H]
        \center
        \includegraphics[width=12cm]{Figuras/regions.png}
        \caption{ exp-26-01-2 (V x Q)}
    \end{figure}



    \subsection{Manual x Automatic Cone Jet stability island maps}

        For validation of the automatic system and classification some experiments were made having both manual and automatic data collecting.

        The Figure 13 shows a printscreen of how the experiment looks like in real time.
        We can see the image generated by the camera in the back.
        The routine code running in pycharm software on the right.
        And also real time signal plottings of the current data on the left.

        \begin{figure}[H]
            \center
            \includegraphics[width=17cm]{Figuras/19:03/axs1.png}
            \caption{running experiment print screen}
        \end{figure}

        In Figure 14 we can see a result of the map generated by the automatic classification in this experiment.

        \begin{figure}[H]
            \center
            \includegraphics[width=10cm]{Figuras/report3/map-exp-26-01.png}
            \caption{ exp-26-01-23 }
        \end{figure}

        Figures 15 and 16 shows that we could achieve a stable cone jet region map with similar shape and values in both manual and automatic classification of the same experiment.

        \begin{multicols}{2}


            \begin{figure}[H]
                \center
                \includegraphics[width=9cm]{Figuras/report3/manual-mapping.png}
                \caption{ exp-26-01 manual classification}
            \end{figure}

            \begin{figure}[H]
                \center
                \includegraphics[width=9cm]{Figuras/report3/map4-stabilityIsland.png}
                \caption{ exp-26-01 automatic classification}
            \end{figure}

        \end{multicols}

        Figures 15 and 16 shows that we could achieve a stable cone jet region map with similar shape and values in both manual and automatic classification of the same experiment.

        Figures 15 and 16 shows that we could achieve a stable cone jet region map with similar shape and values in both manual and automatic classification of the same experiment.


        \begin{multicols}{2}


            \begin{figure}[H]
                \center
                \includegraphics[width=7cm]{Figuras/report4/map7-manual.png}
                \caption{ exp-26-01 manual classification}
            \end{figure}

            \begin{figure}[H]
                \center
                \includegraphics[width=9cm]{Figuras/report4/map7-automatic-line.png}
                \caption{ exp-26-01 automatic classification}
            \end{figure}


        \end{multicols}

        \subsection{Non-dimensional axis}

            \begin{figure}[H]
                \center
                \includegraphics[width=12cm]{Figuras/19:03/non-dimensional-1.png}
                \caption{ exp-26-01-2 (V x Q)}
            \end{figure}

\section{Controller}
\label{sec:controller_results}


    \subsection{Simple Controller}

        \begin{algorithm}
            \caption{simple controller}\label{alg:simple_controller}
            \begin{algorithmic}
            \Function{controller}{$spray\_mode$} 
                
                \If{$spray\_mode$ = $'Intermittent'$ or $spray\_mode$ = $'Dripping'$}
                    \State \Call{send\_voltage\_command}{$voltage$ + 100}
                \ElsIf{$spray\_mode$ = $'Multi Jet'$ or $spray\_mode$ = $'Corona'$}
                    \State \Call{send\_voltage\_command}{$voltage$ - 100}
                \ElsIf{ $spray\_mode$ = $"Cone Jet"$}
                    \Comment{Keep Stable}
                \EndIf

            \EndFunction
            \end{algorithmic}
        \end{algorithm}

        Flowrate perturbation robustness test

        \begin{figure}[H]
            \center
            \includegraphics[width=15cm]{Figuras/19:03/control_first_results.png}
            \caption{ exp-26-01-2 (V x Q)}
        \end{figure}


    \subsection{Robust Controller}

    \subsection{Fuzzy Controller}

        The fuzzy approach of controller was explored and simulated but not used in the final version of the project.
        This because for this fuzzy approach we need to have the input variables for the fuzzy machine to be fuzzified.
        Which means that to use a fuzzy logic in our controller loop the classification must be fuzzyfied and our classification algorithm was not developed in order to give a classification and its current membership function.

        For that, I tried to fuzzyfi the controller input by the data acquired in the step routine. With the data I mapped the area of each spraying mode according to its potential and fuzzyfied this as shown in the Figure X.

        \begin{multicols}{2}
            \begin{figure}[H]
                \centering
                \resizebox{90mm}{!}{\includegraphics{Figuras/fuzzy/fuzzyfy_input.png}}
                \caption{Fuzzyfication}
                \label{fig:fuzzy_input}
            \end{figure}

            \begin{figure}[H]
                \centering
                \resizebox{90mm}{!}{\includegraphics{Figuras/fuzzy/defuzzyfy.png}}
                \caption{Defuzzyfication}
                \label{fig:fuzzy_output}
            \end{figure}
        \end{multicols}

        \begin{figure}[H]
            \centering
            \resizebox{80mm}{!}{\includegraphics{Figuras/fuzzy/rules.png}}
            \caption{Fuzzy Rules}
            \label{fig:fuzzy_rules}
        \end{figure}

        \begin{multicols}{2}

            \begin{figure}[H]
                \centering
                \resizebox{90mm}{!}{\includegraphics{Figuras/fuzzy/test1.png}}
                \caption{Test 1: fuzzy controller}
                \label{fig:fuzzy_test1}
            \end{figure}

            \begin{figure}[H]
                \centering
                \resizebox{90mm}{!}{\includegraphics{Figuras/fuzzy/test2.png}}
                \caption{Test 2: fuzzy controller}
                \label{fig:fuzzy_test2}
            \end{figure}

        \end{multicols}

\section{Atividades do Projeto}
\label{sec:metodo3}

\section {Requisitos do Sistema}
\label{sec:req}


\clearpage