\chapter{Introduction}
\label{chap:intro} %este label será usado para referenciar este capítulo

Electrohydrodynamic Atomization (EHDA) is a way to disintegrate a liquid into droplets by exposing it to a strong electric field.\cite{prunet}
The balance beetween capilary forces and the eletric field on the charged liquid defines the spraying dynamics and droplet size.
The electric current transported by the spray reveals characteristic shapes for different spray modes.
Signal processing techniques can allow a non-visual classification of the spray mode based on the electric current shape.\cite{Sjaaks}
The spray process imposes noise and random sequences on the measured signal making its classification not a trivial task. 
Industrial applications demand automated stabilization of a spray mode. 
This can be achieved by a closed-loop control system. 
This project is about to develop an application that can classify what dynamics the EHDA experiment is current in and control the variables to stabilize in the desired mode. 


\section{Motivation and Justify}
\label{sec:motivacao}
% Argumente sobre a importância do projeto desenvolvido usando uma visão de alto nível, sem entrar em detalhes. Contextualize seu projeto dentro do local de execução ou da literatura e explique como ele é necessário ou inovador. É possível fazer uma breve revisão bibliográfica, confrontando seu trabalho com outras referências bibliográficas para mostrar a sua contribuição. No quesito contribuição, é muito importante deixar claro o tempo todo que partes do projetos foram executadas por outros e que partes foram executadas por você. Caso contrário, corre-se o risco de inadvertidademente tomar crédito pelo trabalho de outrem, o que pode ter implicações legais. 

As pesquisas de EHDA têm contribuído como uma importante ferramenta para
o desenvolvimento da tecnologia . Embora existam aplicações de EHDA em
indústria, a estabilização do modo de pulverização de jato cônico é feita empiricamente e com base em medições de corrente média.
A corrente elétrica que flui transportada pelo spray revela formas características para diferentes modos de atomização.
Essas formas não podem ser simplesmente resumidas por seu valor médio. Na figura um podemos ver um exemplo de cone-jet
modo eletrospray.

Figura 1: exemplo de EHDA

As técnicas de processamento de sinal podem permitir uma classificação não visual do modo de pulverização com base no elétrico
forma atual. O processo de pulverização impõe ruídos e sequências aleatórias no sinal medido tornando-o
classificação não é uma tarefa trivial.
Aplicações industriais exigem estabilização automatizada de um modo de pulverização. Isso pode ser obtido por um sistema fechado
sistema de controle de circuito. A classificação automatizada do modo de pulverização é uma parte crucial de um sistema de controle, assim como
o desenvolvimento de um algoritmo de controle apropriado.

\section{Project objectives}
\label{sec:objetivos}

Tendo em vista o exposto acima, este projeto tem por objetivos:

multipurpose applications
both scientific and industrial application

\begin{enumerate}[a)]
\item 
\item Item 2; 
\item Etc.     
\end{enumerate}

O conteúdo desta seção pode se sobrepor um pouco com o da seção anterior, podendo ela ser um sumário dos pontos expostos anteriormente. A escolha do título da seção talvez seja mais apropriada para a fase de proposta do projeto. Afinal, nesta fase se conhecem os objetivos e não os resultados. Por outro lado, fará pouco sentido discutir objetivos quando o projeto está finalizado, especialmente se tais objetivos não foram alcançados. 


\section{People envolved}
\label{sec:empresa}

% Faça uma breve apresentação da empresa ou laboratório onde o projeto é desenvolvido. Fale um pouco da história da empresa, do mercado em que atua, da sua organização, do departamento em que está inserido o projeto, etc. Descreva também o seu vínculo com a empresa. 
Implementações de processamento de sinal de projetos anteriores
do grupo NHL Stenden Water Technology estão mostrando bons resultados de classificação. Mais pesquisas são
necessário para melhorar a precisão da classificação e pesquisa e implementação de uma classificação adequada
algoritmo. Por causa disso, o trabalho será feito pelo Water Technology Group da NHL Stenden University
de Ciências Aplicadas e em combinação com empresas holandesas para combinar possibilidades de análise com conhecimento
e disponibilidade de infraestrutura.
