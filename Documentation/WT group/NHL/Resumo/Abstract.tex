\addcontentsline{toc}{chapter}{Abstract}

\begin{center}
\huge{{\bf Abstract}}
\vspace{2cm}
\end{center}

    Electrohydrodynamic Atomization (EHDA), also called electrospray, is a liquid atomization technique that
    produces micro- and nanometric charged droplets within a narrow size distribution by using high electric fields (kV/cm).
    According to Cloupeau and Prunet-Foch\cite{prunet} (1994), electrosprays can generate droplets in different ways, which the authors
    named "electrospray modes". These modes may be adjusted by varying the strength of the electric field and flow rate,
    but also depend on liquid properties and system geometry. In their work, the authors proposed four possible EHDA
    modes: dripping, intermittent, cone-jet and multi-jet, which are generally distinguished visually. Verdoold et al.\cite{Sjaaks} (2014)
    recently suggested a classification approach based on the behavior of the electric current of the electrospray process.
    
    This project develops a closed-loop control method for EHDA devices that uses real-time, electric current-based (hence
    non-visual) spray mode classification.
    The proposed electrospray system is entirely automatic, where all the peripherals, such as HV power supply and syringe
    pump, are controlled by a computer which executes their routines.
    The system classifies spray mode dynamics using real-time current data and changes EHDA operating parameters such
    as liquid flowrate and applied voltage to achieve and maintain the chosen spray mode. The electrospray modes are
    validated in real time by using a high-speed camera.
    As compared to conventional manual approaches, the implemented control algorithm achieves higher
    accuracy and lower transient time. Therefore, a completely autonomous EHDA system opens the door to potential
    industrial applications. In addition, the use of the electric current signal will be useful to further study electrospray
    processes, leading to better control on droplet generation (frequency, size and charge). The incorporation of Machine
    Learning to improve mode categorization will be a future development.
 
